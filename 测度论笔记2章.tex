\chapter{测度}

\section{测度和外测度}

我们通过先定义外测度、再定义外测度的适用范围(可测集族)的方式定义真正的测度。

如果非负集函数$\mu$正则(把空集映射到0)且(在集合族上)可数可加,则称$\mu$为(集合族上的)测度。此外还定义有限测度和$\sigma$有限测度(如果有限测度的集合的可数并覆盖待测集合;这个条件比有限测度宽松一些,譬如实数$\RR$上的lebesgue测度)。

任何一个测度都具有有限可加性和可减性(被减数不能为无穷)。

\ls{例(计数测度):把集合映射到集合的基数。如果至多是可数无穷,那么就是$\sigma$测度;如果至多是有限集,那么就算有限测度。}

\ls{例(点测度):当集合包含一个指定点时取1,否则取0的测度。是有限测度。它的线性组合也是有限测度,称作点测度。}

\ls{例(区间长度):区间的长度也可以作为有限开闭区间族测度。}

\ls{例(不具备可数可加性):对于一个数列,给有限子列和子列指派测度。如果$A$是有限集,那么测度为0;如果$A^c$为有限集,那么测度为无穷。这没有可数可加性,因为有限集之可数并可能是子列,测度无穷。但是,有有限可加性。}

\ls{定理(分布函数是测度):对于$\RR$上递增、右连续的实值函数$F$,对有限开闭集族,$F$在区间上的差是测度。}

证明:只需证明可数可加性。1.证明刚好包含时的有限可加【显然】。2.内-可数逼近。3.外-有限逼近。4.整合结论。

1.【显然】有限可加性。记$\miu(a,b]=(F(b)-F(a))I_{a<b}$。设$n$个不交的开闭区间的并恰好是$(a,b]$。证明有限可加。

2.设不交开闭区间的可数并包含于$(a,b]$之内,证明小区间测度的和小于等于大区间的测度。排好次序,固定n进行证明即可。

3.设对每个$n$,不交开闭区间的有限并包含$(a,b]$,证明小区间测度的部分和大于等于大区间的测度。也很符合直觉,使用归纳法即可证明。

4.设不交开闭区间的可数并等于$(a,b]$,证明可数可加。由第2问,只需要证明小区间测度的和大于等于大区间的测度。

首先把$(a,b]$\zd{变成渐近态(\zd{开区间$\yd$闭区间})}$[a+\ita,b]$,然后把许多个($\eps-2^n$技术)开闭覆盖的闭端也同样开化(\zd{闭区间$\yd$开区间})。然后利用有限覆盖定理和第3问(此时又把每个开化的闭端闭回去)。这样就带着2个无穷小量$\renyi\ita,\eps>0$完成了证明。最后令它们都趋于0即可。

\ls{(建立了概率)\duan 其实我们主要关心的不是一般的测度,是$\sigma$域上的测度。以后称$(X,\Fh,\miu)$为测度空间。如果全集的测度为1则称为概率空间。概率空间中的可测集合称为事件,测度函数称为概率测度,简称概率。}

\ls{例:(可数测度)可数集合(默认幂集为可测集族)到非负实数列的映射是一个测度。}

\ls{例:(古典概型)在有限集上定义计数测度并归一化后的映射是测度,该概率空间称为古典概型空间。}

在一般的$\sigma$域上建立具体的测度要复杂得多。就像实变函数一样,通常在区间上建立测度。为了使证明更简单,通常在开闭区间族(半环)上定义测度。然后,把它扩展到它张成的$\sigma$域上去。





